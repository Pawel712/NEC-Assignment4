\documentclass[12pt]{report}
\usepackage[utf8]{inputenc}
\usepackage{graphicx}
\usepackage{amsmath}
\usepackage{hyperref}
\usepackage{makecell}
\usepackage{adjustbox}
\usepackage{geometry} 
\usepackage{subcaption}
\usepackage{float}

\title{Assignment 4: \\ Optimization with Genetic Algorithms \\ Neuronal and evolutionary computing\\ 2023 - 2024}
\author{Pawel Puzdrowski}
\date{2024-01-XX}

\begin{document}
    \maketitle
    \tableofcontents
    \newpage
    \section{1.}
    Selection methods: \\
    \begin{itemize}
        \item Fitness proportionate selection 
        \item Tournament selection
    \end{itemize}
    Crossover methods: \\
    \begin{itemize}
        \item One point crossover
        \item Uniform crossover
    \end{itemize}
    Mutation methods: \\
    \begin{itemize}
        \item One bit flip mutation
        \item Scramble mutation
    \end{itemize}
    Nästa gång:\\
    Beskriva varje metod för selection/crossover/mutation med beskrivning och bild, blir enklare att förklare på redovisning\\
    \\
    Hitta källa varifrån jag har strukturen av genetic algorithm

    \newpage
    \section{2.}
    
    %\bibliographystyle{plainurl} 
	%\bibliography{ref} 

    %%  Look into what lambda does in the code

    % From this site: https://towardsdatascience.com/evolution-of-a-salesman-a-complete-genetic-algorithm-tutorial-for-python-6fe5d2b3ca35
    % I got the idea how to plot the improvements in the algorithm

    % From this site: https://towardsdatascience.com/evolution-of-a-salesman-a-complete-genetic-algorithm-tutorial-for-python-6fe5d2b3ca35
    % I got the info about the different methods for selection

    % From this site: https://www.tutorialspoint.com/genetic_algorithms/genetic_algorithms_mutation.htm
    % I got the info about the different methods for mutation

    % from this site: https://www.tutorialspoint.com/genetic_algorithms/genetic_algorithms_crossover.htm
    % I got the info about the different methods for crossover



    % jag har tagit dataset a280.tsp burma14.tsp, ch150.tsp från:
    % http://comopt.ifi.uni-heidelberg.de/software/TSPLIB95/tsp/

    % jag har tagit dataset att48.tsp från:
    % https://people.sc.fsu.edu/~jburkardt/datasets/tsp/tsp.html
    
    % smallDataset.tsp är en modifierad version av att48.tsp eftersom behövde ha en liten dataset för testning



\end{document}